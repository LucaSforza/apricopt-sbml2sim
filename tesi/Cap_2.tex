\chapter{Modellazione Biologica}

\section{Introduzione}

I modelli biologici che consideriamo sono sistemi dinamici tempo-invarianti, generalmente non lineari, descritti da equazioni differenziali ordinarie (ODE).

Lo stato del sistema al tempo \(t\) è un vettore \(x(t) = (x_1(t),\dots,x_n(t))^\top \in [0,1]^n \subseteq \mathbb{R}^n\), dove \(x_i(t)\) rappresenta la concentrazione normalizzata della i-esima specie. Poiché le concentrazioni sono normalizzate, \(x_i(t)\in[0,1]\) per ogni \(i\) e ogni \(t\).

In forma compatta le ODE si scrivono
\[
\frac{dx}{dt} = f(t,x).
\]

Sia \(m\) il numero di reazioni nel modello.

Per ogni specie \(x_i\) definiamo
$R_{x_i} \subseteq \{1,\dots,m\}\quad\text{(reazioni in cui \(x_i\) è reattante)}$,
$P_{x_i} \subseteq \{1,\dots,m\}\quad\text{(reazioni che producono \(x_i\))}.$

Se \(R_{x_i}\neq\emptyset\) e \(P_{x_i}\neq\emptyset\), l'evoluzione temporale di \(x_i\) è data da:
\[
\frac{dx_i}{dt}=\sum_{j\in P_{x_i}} v_j(t) - \sum_{j\in R_{x_i}} v_j(t),
\]
dove \(v_j(t)\) è la velocità della reazione \(j\).

La velocità di una reazione è data dalla legge dell'azione di massa (mass action).

La velocità di una reazione chimica è proporzionale al prodotto delle concentrazioni dei reagenti, ciascuna elevata al proprio coefficiente stechiometrico

La velocità di una reazione è influenzata da modificatori (enzimi, inibitori). Per modellare l'effetto dei modificatori si usa spesso la funzione di Hill. Per ogni specie modificatrice \(x_\ell\) introduciamo una funzione di Hill generica
\[
H_\ell(t) =
\begin{cases}
\dfrac{x_\ell(t)^{10}}{K_\ell + x_\ell(t)^{10}}, &\text{se \(x_\ell\) agisce da attivatore},\\[8pt]
\dfrac{K_\ell}{K_\ell + x_\ell(t)^{10}}, &\text{se \(x_\ell\) agisce da inibitore},
\end{cases}
\]
dove \(K_\ell>0\) è la costante di mezzo-saturazione e dato che è ingnota viene aggiunta cime parametro del modello.

Per la reazione \(i\in\{1,\dots,m\}\) definiamo inoltre:


$S_i=\{j\in\{1,\dots,n\}\mid x_j \text{ è reattante della reazione } i\}$,


$M_i=\{j\in\{1,\dots,n\}\mid x_j \text{ è modificatore della reazione } i\}$.

$n_i^j$ è la stochiometria della specie $j$ nella reazione $i$.

La velocità \(v_i(t)\) si esprime quindi come
\[
v_i(t)=k_i\;\prod_{j\in S_i} x_j(t)^{n_i^j}\;\prod_{j\in M_i} H_j(t),
\]
dove \(k_i>0\) è la costante cinetica della reazione. Essa essendo sconosciuta viene aggiunta
come parametri del modello.

\section{modelli biologici di Pathway}

Un pathway metabolico (o via metabolica)  è una sequenza di reazioni biochimiche che collega metaboliti chiave.

Un pathway è formato da specie di input che non vengono prodotte da reazioni, ma vengono immesse nel pathway.
Specie intermedie che sono sia prodotte che consumate da reazioni, specie di output che sono solo prodotte da reazioni e mai consumate
e specie che non vengono ne prodotte ne consumate da alcuna reazione, ma sono enzimi o inibitori che influscono nelle altre reazioni.

Un esempio tipico di input è l'ATP che funge da "energia" per molte reazioni chimiche.

Esse non vengono prodotte da reazioni (\(P_{x_i}=\emptyset\)) ma vengono immesse nel sistema; si modellano con un termine di ingresso constante \(k_{\mathrm{in},i}\):

\[
\frac{dx_i}{dt}=k_{\mathrm{in},i} - \sum_{j\in R_{x_i}} v_j(t).
\]

I pathway hanno degli output che sono specie che vengono prodotte ma non consumate (\(R_{x_i}=\emptyset\)); ad esse si aggiunge una reazione di degradazione, che avrà una constante cinetica e ha come unico reattante
la concentrazione dell'output stesso, quindi seguendo la mass action rule l'equazione differenziale per gli output è come segue:
\[
\frac{dx_i}{dt}=\sum_{j\in P_{x_i}} v_j(t) - k_{\mathrm{out},i}\,x_i(t),
\]

Nei modelli biologici di pathway possono essere presenti specie che non partecipano ne come reattanti ne come prodotti alle reazioni, ma solo come
modificatori. Quindi queste specie devono avere una costante di immissione ed anche una di degradazione.

Queste specie sono enzimi o inibitori esterni, quindi influiscono sulla velocità delle altre reazioni del modello. La loro dinamica è descritta come segue:
\[
\frac{dx_i}{dt}=k_{\mathrm{in},i}-k_{\mathrm{out},i}\,x_i(t).
\]

